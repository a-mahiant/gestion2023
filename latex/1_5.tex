\documentclass[a4paper,12pt]{article}

%% NOTE POUR ALICE %%
% j'ai mis en commentaire l'appel au package hyperref + l'inclusion des 2 images (pour page de garde)
% pour pouvoir générer un pdf de travail: à réactiver lors de la mise en commun ^^

%%%%PAGE SETTINGS%%%%
	\usepackage[utf8]{inputenc} 
	\usepackage[top=2cm,bottom=3cm]{geometry} 
		
%%%%PREAMBLE%%%%
	\usepackage{graphicx} 
	\usepackage[round]{natbib} % gestion des citations
	\usepackage[french]{babel} 
	\usepackage{titlesec}
	\usepackage[hyphens]{url} 
	% \usepackage[pdftex,urlcolor=black,colorlinks=true,linkcolor=black,citecolor=black]{hyperref} 
	\usepackage{comment}
	
	\titleformat{\section}{}{}{0em}{\bf\LARGE}
	\titleformat{\subsection}{}{}{0em}{\bf\Large}
	\titleformat{\subsubsection}{}{}{0em}{\bf\normalsize}

%%%%FONT%%%%
	\usepackage{lmodern}
	\renewcommand*\familydefault{\sfdefault}
	\usepackage[T1]{fontenc}
	\usepackage[bottom]{footmisc}
	\setlength{\parskip}{1em}
	\interfootnotelinepenalty=10000
	
	\bibliography{Bibliography_exam_2021_2022_Q1} 
	\bibliographystyle{ieeetr}
	
	\title{titre} 
	\author{Alice Mahiant, Guillaume Quintin, Chloé Steylaers} 
	\date{2022,}

%%%%DOCUMENT%%%%

\setcounter{tocdepth}{6}
\begin{document}
\begin{center}
% \includegraphics[width=0.6\textwidth]{logoulb.jpg}
% \includegraphics[width=0.1\textwidth]{image.png}
 \end{center}
 \vspace*{1cm}
 Master Sciences et Technologies de l'Information et de la Communication
 \newline STIC-B500 - Projet : gestion et aspects méthodologiques
   \vspace*{0,5cm}
    \newline Enseignante: ROSSION Françoise
 \newline 1ère session  - Janvier 2023.
  \vspace*{0,5cm}
 \newline \rule{11cm}{0,02cm}
 \vspace*{1,5cm}
 \newline \textbf{{\Large Ecriture du cahier des charges pour la révision du site internet des archives de l'Etat qui permet d'y faire des recherches (https://search.arch.be/fr/)}}
   \vspace*{0,5cm}
 \newline\textbf{{\large Travail de groupe.}}
 \vspace*{1cm}
 \newline MAHIANT Alice
 \newline QUINTIN Guillaume
 \newline STEYLAERS Chloé
 \vspace*{0,5cm}
 \newline \rule{4cm}{0,02cm}
  
%%%%%TABLE OF CONTENTS%%%%
\begin{footnotesize}
\newpage
\tableofcontents
\newpage
\end{footnotesize}

\begin{comment}
%%%%*MODALITES
\begin{scriptsize}
\section{Modalités}
texte
\newline
	\begin{itemize}
	\setlength{\itemsep}{5pt}
		\item{texte ;}
		\item{texte ;}
		\item{texte.}
	\end{itemize}
 \vspace*{0,3cm}
texte
\newline texte.
	\subsection{subsection}
	Texte.
	\subsection{subsection}
	texte
	\begin{itemize}
	\item{texte}
	\item{texte}
	\item{texte}
	\end{itemize}
	texte.
	\subsection{subsection}
	\begin{itemize}
	\item{texte}
		\begin{itemize}
		\item{texte}
		\end{itemize}
	\item{texte}
		\begin{itemize}
		\item{texte}
		\item{texte}
		\end{itemize}
	\end{itemize}
	\subsection{subsection}
	Texte.
\end{scriptsize}
%%%%


\newpage
\end{comment}

%%%%SECTION 1 - introduction%%%%


\section{Section 1 - Introduction}



\section{Section 2 - Description du projet}



\subsection{Périmètre}



\subsection{Typologie des utilisateurs cibles}



\subsection{Objectifs SMART}



\section{Section 3 - Analyse SWOT}



\section{Section 4 - Gouvernance}



\section{Section 5 - Planning général : WBS et diagramme de GANTT}



\section{Section 5 - Gestion des risques}



\end{document}
