\documentclass[a4paper,12pt]{article}

%% NOTE POUR ALICE %%
% j'ai simplifié l'appel au package hyperref + l'inclusion des 2 images (pour page de garde)
% pour pouvoir générer un pdf de travail: à remettre lors de la mise en commun ^^

%%%%PAGE SETTINGS%%%%
	\usepackage[utf8]{inputenc} 
	\usepackage[top=2cm,bottom=3cm]{geometry} 
		
%%%%PREAMBLE%%%%
	\usepackage{graphicx} 
	\usepackage[round]{natbib} % gestion des citations
	\usepackage[french]{babel} 
	\usepackage{titlesec}
	\usepackage[hyphens]{url} 
	
	%ce package a été simplifié!!
	\usepackage{hyperref} 
	
	\usepackage{comment}
	
	\titleformat{\section}{}{}{0em}{\bf\LARGE}
	\titleformat{\subsection}{}{}{0em}{\bf\Large}
	\titleformat{\subsubsection}{}{}{0em}{\bf\normalsize}

%%%%FONT%%%%
	\usepackage{lmodern}
	\renewcommand*\familydefault{\sfdefault}
	\usepackage[T1]{fontenc}
	\usepackage[bottom]{footmisc}
	\setlength{\parskip}{1em}
	\interfootnotelinepenalty=10000
	
	\bibliography{Bibliography_exam_2021_2022_Q1} 
	\bibliographystyle{ieeetr}
	
	\title{titre} 
	\author{Alice Mahiant, Guillaume Quintin, Chloé Steylaers} 
	\date{2022,}

%%%%DOCUMENT%%%%

\setcounter{tocdepth}{6}
\begin{document}
\begin{center}
% \includegraphics[width=0.6\textwidth]{logoulb.jpg}
% \includegraphics[width=0.1\textwidth]{image.png}
 \end{center}
 \vspace*{1cm}
 Master Sciences et Technologies de l'Information et de la Communication
 \newline STIC-B500 - Projet : gestion et aspects méthodologiques
   \vspace*{0,5cm}
    \newline Enseignante: ROSSION Françoise
 \newline 1ère session  - Janvier 2023.
  \vspace*{0,5cm}
 \newline \rule{11cm}{0,02cm}
 \vspace*{1,5cm}
 \newline \textbf{{\Large Ecriture du cahier des charges pour la révision du site internet des archives de l'Etat qui permet d'y faire des recherches (https://search.arch.be/fr/)}}
   \vspace*{0,5cm}
 \newline\textbf{{\large Travail de groupe.}}
 \vspace*{1cm}
 \newline MAHIANT Alice
 \newline QUINTIN Guillaume
 \newline STEYLAERS Chloé
 \vspace*{0,5cm}
 \newline \rule{4cm}{0,02cm}
  
%%%%%TABLE OF CONTENTS%%%%
\begin{footnotesize}
\newpage
\tableofcontents
\newpage
\end{footnotesize}

\begin{comment}
%%%%*MODALITES
\begin{scriptsize}
\section{Modalités}
texte
\newline
	\begin{itemize}
	\setlength{\itemsep}{5pt}
		\item{texte ;}
		\item{texte ;}
		\item{texte.}
	\end{itemize}
 \vspace*{0,3cm}
texte
\newline texte.
	\subsection{subsection}
	Texte.
	\subsection{subsection}
	texte
	\begin{itemize}
	\item{texte}
	\item{texte}
	\item{texte}
	\end{itemize}
	texte.
	\subsection{subsection}
	\begin{itemize}
	\item{texte}
		\begin{itemize}
		\item{texte}
		\end{itemize}
	\item{texte}
		\begin{itemize}
		\item{texte}
		\item{texte}
		\end{itemize}
	\end{itemize}
	\subsection{subsection}
	Texte.
\end{scriptsize}
%%%%


\newpage
\end{comment}

\section{Section 1 - Introduction}

Dans le cadre de ce travail, nous considérons une éventuelle refonte du site internet \href{https://search.arch.be/fr/}{La recherche aux Archives de l'État}. Comme son nom l'indique, il permet d'effectuer en ligne des recherches au sein des collections gardées par les Archives de l'État (que nous abrégerons ci-après \frquote{AE}). Notre tâche, plus modeste, s'imbrique dans ce projet de grande envergure : nous devons mener une récolte et une analyse des besoins qu'éprouvent les utilisateurs vis-à-vis de ce site de recherche.

Puisque notre tâche ne peut se penser sans l'ancrer dans le projet de refonte, notre travail s'est construit sur base d'une réflexion itérative, alternant entre une vision à large échelle de la refonte dans sa globalité et un travail de précision pour l'analyse des besoins à proprement parler. Ainsi, nous commençons ce travail par une description du projet de refonte à une échelle macroscopique. Ensuite, nous concevons une récolte et une analyse des besoins dans le détail, ce qui inclut une analyse SWOT, une matrice de gouvernance RACI, un planning général (WBS et GANTT), une gestion des risques et une priorisation des besoins fonctionnels identifiés selon une matrice MoSCoW. Nous continuons avec un accompagnement du changement, un plan de communication et des recommandations pour des méthodes de gestion de projet, pour lesquels nous considérons à nouveau le projet de refonte du site. Enfin, nous achevons ce travail par un retour réflexif sur notre propre expérience tout au long de ce quadrimestre.

\section{Section 2 - Description du projet}

Le périmètre du projet est la révision du site internet \frquote{\href{https://search.arch.be/fr/}{La recherche aux Archives de l'État}}, dans le but qu'il réponde davantage aux besoins de ses utilisateurs. Plus précisément, nous devons nous préparer la réalisation d'un cahier des charges pour une telle révision. Pour ce faire, nous devons réaliser une analyse métier et une analyse fonctionnelle du projet de révision (\textit{scope in}). À l'inverse, l'analyse technique - bien que pertinente - se situe en dehors du périmètre de notre projet (\textit{scope out}) : en effet, nous ne disposons pas des compétences ou des accès nécessaires pour la mener à bien.

Afin de mieux cerner les utilisateurs du site internet, qui sont au centre de notre travail, nous les avons classés en quatre catégories : 
\begin{itemize}
	\item Les notaires et les géomètres;
	\item Les académiques : étudiants, chercheurs, enseignants ;
	\item Les fonctionnaires ;
	\item Les généalogistes.
\end{itemize}
Cette classification, qui épouse peu ou prou celle qui existe déjà que le site internet, permet d'identifier les besoins spécifiques à l'une ou l'autre groupe d'utilisateurs, même si certains besoins sont communs à plusieurs catégories.

\subsection{Résultats de l'analyse Mareva en quelques mots}

Pour aborder la description du projet de refonte du site internet, nous avons décidé d'en réaliser une analyse Mareva. Même si une telle analyse exige des données dont nous ne disposons pas (comme un budget) et que par conséquent elle s'est révélée incomplète, il nous a semblé pertinent de la résumer pour en extraire les éléments intéressants.

Avant toute analyse Mareva, il est nécessaire de récolter tout un ensemble d'informations. Le périmètre du projet a été établi ci-dessus, tandis que la durée ou le budget prévisionnels de celui-ci sont difficiles à estimer avec nos seules connaissances (limitées) en la matière. L'administration concernée par ce changement, ce sont les AE, en particulier son équipe IT. Le nombre de personnes concernées est théoriquement l'ensemble de la population belge, concrètement les personnes consultant le site internet (nombre qui pourrait augmenter avec un site plus intuitif). Les impacts positifs du projet seraient, du côté des AE, une plus grande autonomie des utilisateurs et une meilleure visibilité et, du côté des utilisateurs, un gain de temps, d'énergie et d'efficacité de leurs recherches. Toutefois, une refonte demanderait autant aux AE qu'aux utilisateurs de se réhabituer à un nouveau système et ajouterait une dépense fixe au budget des AE (maintien du site internet révisé), sans considérer le budget qui serait dédié à la refonte elle-même. Même s'il est difficile d'en évaluer quantitativement les coûts, il est aisé de décomposer ces derniers en catégories : coûts de développement, d'achats de nouveaux logiciels, de communication sur le nouveau système, de formation au nouveau système...

L'analyse Mareva en elle-même est divisée en cinq étapes. Les résultats obtenus à chacune de ces étapes sont les suivants :
\begin{enumerate}
	\item Rentabilité : Même si la refonte du site internet représente un coût certain, elle permet un gain en efficacité indéniable pour les AE si l'outil de recherche est efficace et évite un recours à l'humain (qui mobilise des ressources humaines). Ainsi, il est possible de quantifier les gains en additionnant les ressources humaines qui peuvent être redéployées dans d'autres secteurs des AE avec les éventuelles ressources financières économisées (par une rationalisation du site internet lors de la refonte) ;
	\item Maîtrise du risque : Les risques liés au projet peuvent être divisés en quatre catégories : les risques liées de déroulement, les risques juridiques, les risques techniques et les risques de déploiement. Dans son ensemble, le projet ne présente pas de risques particulièrement élevés. Voici les observations liées à chaque catégorie de risques :
	\begin{itemize}
		\item Déroulement : Ces risques sont faibles. En effet, les objectifs et le périmètre du projet sont clairs, le planning n'est pas serré et seules les AE sont concernées. Il est toutefois nécessaire de surveiller le budget qui, lui, n'est pas illimité ;
		\item Juridiques : Ces risques sont également faibles. La refonte ne répond pas à un changement de réglementation. De nombreuses réglementations encadrent les archives (dont le RGPD), mais la refonte du site n'introduit aucun changement ;
		\item Techniques : Ces risques sont difficiles à évaluer, puisque nous n'avons pas accès à la structure technique du site internet, mais nous pouvons les considérer comme modérés. En effet, bien que la refonte nécessite l'introduction de techniques et/ou de logiciels nouveaux, la complexité technique d'une telle entreprise est faible et s'appuie sur de l'expérience existante dans l'institution ou dans des institutions s\oe{}urs. Il faut toutefois garantir la pérennité des données si elles sont transférées, car une quelconque perte serait catastrophique ;
		\item Déploiement : Ces risques sont les plus prononcés. Même si l'on peut considérer que les ressources nécessaire au changement seront à la disposition des AE, cette refonte concerne les processus actuellement en place dans l'institution et risque d'interrompre momentanément le service fourni aux utilisateurs. De plus, un accompagnement efficace au changement doit absolument être mis en place lors de la refonte, tant en interne qu'en externe. Une adhésion à la refonte par le public et par l'institution est nécessaire pour mener celle-ci à terme ;
	\end{itemize}
	\item Externalité : Le projet de refonte possède très clairement une externalité, puisqu'il vise à proposer aux utilisateurs un moteur de recherche plus efficace et intuitif que l'actuel. Qualitativement, l'objectif est de réduire le nombre d'interlocuteurs nécessaires pour trouver l'information recherchée (dans l'idéal, aucun), de proposer un haut niveau de personnalisation du service selon les types d'utilisateurs identifiés. Le service serait proposé autant en ligne que dans les salles de lecture des AE. La refonte participe hautement à la promotion de la société de l'information, puisqu'elle garantit le droit à l'information des citoyens et souligne les améliorations apportées par les nouvelles technologies en la matière. Elle permet également d'assurer une transparence de l'information et une cohérence sociale que prodigue l'accès facilité aux archives. Quantitativement, le service s'adresse en principe à l'ensemble de la population, concrètement seulement à une partie infime de celle-ci. Il est permis d'espérer que la refonte permettrait d'augmenter cette partie. Le gain temporel doit également être mentionné, puisque l'information serait plus intuitivement trouvable. En somme, l'externalité du projet obtient un résultat très haut et est l'un des points forts du projet de refonte ;
	\item Internalité : L'internalité du projet est moins prononcée que son externalité, bien que tout de même modérément positive. En effet, la qualité de vie du personnel des AE augmenterait certainement, puisqu'une partie des questions liées à l'utilisation du site internet ne leur arriverait plus. Néanmoins, à l'exception de ce point, les critères de l'analyse Marena ne sont pas pertinentes dans le cadre de notre projet ;
	\item Nécessité : La refonte du site de recherche est nécessaire. Elle permettrait d'augmenter l'efficience de l'action publique en améliorant une situation existante, mais c'est surtout une lourde obligation externe qui repose sur le projet. L'accès aux archives est un droit que garantit la législation belge et il est de la responsabilité des AE d'assurer un accès le plus aisé possible. Cet accès aux archives est également une obligation politique, liée au droit à l'information et à la promotion de la société de l'information précédemment mentionnée.
\end{enumerate}

\subsection{Objectifs SMART}

Nous avons conçu quelques objectifs SMART permettant de suivre et de quantifier, tels des indicateurs de performance de notre projet, l'évolution et le succès de l'amélioration du site web des AE :

\begin{itemize}
	\item Le cahier des charges de la révision du site web et de son moteur de recherche doit être livré endéans un an ;
	\item L’objectif est d’avoir un échantillon représentatif des utilisateurs actuels et potentiels du site web (administration, historiens, généalogistes, etc.) nécessaire à la réalisation du cahier des charges ;
	\item Les enquêtes de satisfaction sur la refonte du site réalisées pendant les trois mois suivants la mise à disposition des utilisateurs finaux du nouveau site sont positives à 80 \% ;
	\item Lors de la phase de test du produit, au moins 95 \% des utilisateurs doivent être satisfaits de leurs recherches ;
	\item L’ensemble des interviews programmées pour l’analyse des besoins doivent être effectuées durant une période de trois mois (c’est-à-dire une période lors de laquelle le site demeure inchangé) ;
	\item Une fois que la solution est déployée, l’objectif est de récolter pendant deux mois les retours des utilisateurs interviewés au début du projet dans le but d’évaluer l'intuitivité du site.
	
\end{itemize}

\section{Section 3 - Analyse SWOT}

Avant de nous plonger dans la planification et la réalisation de l'analyse des besoins, il fut pertinent de considérer les facteurs positifs et négatifs, qu'ils soient internes ou externes, qui influençerait celle-ci. En d'autres termes, d'effectuer une analyse SWOT de notre projet.

Les facteurs positifs internes (\textit{Strengths}) : 
\begin{itemize}
	\item Les backgrounds des chef.fe.s du projet sont différents et complémentaires, avec un pied dans le monde de l'histoire ;
	\item Chacun des trois chef.fe.s du projet ont suivi une formation en archivistique ;
	\item Les contacts avec certains types d'utilisateurs (les académiques) sont facilités ;
	\item Des réunions en présentiel sont aisément réalisables ;
	\item Les chef.fe.s du projet communiquent bien entre eux et ont l'habitude de travailler ensemble.
\end{itemize}

Les facteurs négatifs internes (\textit{Weaknesses}) : 
\begin{itemize}
	\item D'autres types d'utilisateurs (généalogistes, administration) sont plus difficilement accessibles ;
	\item Les chef.fe.s du projet n'ont que très peu d'expérience dans la gestion de projet ;
	\item Au vu de ce manque d'expérience, l'interprétation de certaines consignes pourrait être erronée.
\end{itemize}

Les facteurs positifs externes (\textit{Opportunities}) : 
\begin{itemize}
	\item Les utilisateurs finaux sont motivés et enthousiastes ;
	\item Il existe une volonté de refonte du site aux AE, et par conséquent une ouverture au dialogue ;
	\item Le projet et sa nécessité sont légitimes et légitimées tant par les utilisateurs que l'institution.
\end{itemize}

Les facteurs négatifs externes (\textit{Threats}) : 
\begin{itemize}
	\item Nécessité de discrétion (il n'est pas possible de contacter directement les AE) ;
	\item Certains utilisateurs pourraient ne pas être totalement honnêtes ou complets dans leurs interviews pour ne pas vexer les AE ;
	\item Les utilisateurs pourraient ne pas avoir le temps de nous recevoir ;
	\item Les utilisateurs pourraient ne pas nous répondre.
\end{itemize}

\section{Section 4 - Gouvernance}

Pour établir la gouvernance de notre projet, nous avons utilisé une matrice RACI. Cette dernière, reproduite ci-après, prend la forme d'un tableau à double entrée. Les lignes correspondent aux tâches qui composent le projet (et qui font écho à celles qui se trouvent dans le WBS), tandis que les colonnes correspondent aux acteurs du projet.

En l'occurrence, le projet en compte quatre : la direction (l'enseignante), les chef.fe.s de projet (le groupe étudiant), l'équipe projet (le groupe étudiant) et les utilisateurs (les personnes interviewées). Nous avons choisi de considérer que le groupe étudiant occupait tantôt le rôle de chef.fe.s de projet, tantôt celui de membre de l'équipe projet. En effet, nous nous chargions autant de la gestion de notre projet que de sa réalisation et pour produire une matrice RACI pertinente, il était nécessaire de bien distinguer les deux rôles même si ces derniers étaient tenus par les mêmes personnes.

% insérer matrice RACI

\section{Section 5 - Planning général}

\subsection{WBS}

Pour planifier la récolte des besoins, nous l'avons divisée selon le principe du Work Breakdown Structure (WBS). Nous avons identifié sept phases séquentielles, chacune divisée en tâches plus précises. Ce WBS est détaillée dans le tableau suivant :
% insérer tableau WBS

\subsection{Diagramme de GANTT}

Nous avons ensuite placé les phases et tâches identifiées dans le WBS sur un diagramme de GANTT. Même si la structure de notre projet est principalement séquentielle, le diagramme de GANTT permet de mettre en évidence les quelques tâches qui peuvent s'effectuer en parallèle (en particulier les interviews).
% insérer le diagramme

\section{Section 6 - Gestion des risques}
Nous avons identifié les risques, en grande majorité de nature humaine, qui menacent notre projet de récolte des besoins et les avons classés selon leur niveau de gravité. Ils sont repris dans le tableau suivant, accompagnés d'un plan d'action (actions préventives et correctrices) :
% insérer le tableau risques

\section{Section 7 - Analyse des besoins métier et solutions fonctionnelles}
\subsection{La méthode employée}
Nous avons privilégié une méthode qualitative. Nous avons élaboré un questionnaire qui se découpe en trois axes. Le premier interroge la façon dont les utilisateurs exploitent actuellement le site et le moteur de recherche ; le second axe se concentre sur les différents problèmes que les utilisateurs ont pu rencontrer et enfin, le troisième et dernier axe concerne les éventuelles améliorations que les utilisateurs souhaiteraient avoir. Nous envoyions à l’avance trois grandes questions relatives à ces axes pour que l’utilisateur puisse réfléchir en avance à son expérience utilisateur. 
%insérer les questions envoyées à l'avance


\subsection{Le profil des utilisateurs}
Le public que nous avons pu atteindre se compose essentiellement d'académiques concernés par la question (M. De Valeriola, M. Tallier) et des étudiants en histoire. Les utilisateurs “internes” et “particuliers” sont plus difficiles à interviewer, car on ne peut pas contacter les AGR. Nous avons en outre pu contacter une association de généalogistes successoraux (notons cependant que la généalogiste que nous avons pu interroger fut également une ancienne étudiante en histoire).
Tous les utilisateurs sont un public à haut intérêt, mais avec une basse influence dans le système des AGR. Observer les utilisateurs directement est compliqué vu qu’il s’agit d’une démarche numérique. 
Nous notons cependant que, si nous avions pu inclure les AE dans le cadre de ce travail, il aurait pu être intéressant d'utiliser, en complément de la méthode qualitative, un questionnaire quantitatif.
%insérer le questionnaire détaillé

\subsection{Les problèmes et contraintes rencontrées de la situation actuelle}
Au terme de nos entretiens, il nous est apparu que les utilisateurs rencontraient tous certains problèmes de façon récurrente. 
Premièrement, les résultats aux recherches produisent du silence : en utilisant un mot-clef x, les documents traitant d’une thématique similaire à x mais qui ne contiendraient pas le même mot-clef n’apparaissent pas dans les résultats. En outre, la prise en compte des synonymes dans la recherche n'apparait pas efficiente. Les utilisateurs doivent utiliser le bon mot-clef sous lequel le document a été encodé, sous peine sinon de ne pas trouver le dit document. En outre, les utilisateurs ont soulevé qu'il n’y ait aucune prise en compte du caractère multilingue des archives, nécessitant de mettre les bons mots-clés dans les différentes langues pour avoir les résultats. De même, la variance orthographique (par exemple des noms de famille) n’est pas prise en compte, ce qui peut encore diminuer le nombre de résultats pertinents de la requête. 

Deuxièmement, les utilisateurs ont tous signalé le problème du bruit dans les résultats de recherches, c'est-à-dire que les résultats n'ont pas été jugés pertinents par les utilisateurs (exemple : recherche d'archives sur les comptes urbains au Moyen Âge qui renvoie comme résultat de recherche des documents ou fonds concernant des archives judiciaires du 19e siècle, etc.). Les utilisateurs tentent de contrer ce problème grâce aux différents filtres proposés, mais ces derniers ne sont pas toujours pleinement satisfaisants. 

Troisièmement, les utilisateurs ont soulevé que le site ne leur paraissait pas très intuitif, et certains utilisateurs ont le sentiment de se perdre dans les méandres de l'arborescence archivistique. Une suggestion récurrente que nous ont faite tous les utilisateurs est la mise en place d'un meilleur "mode d'emploi" du moteur de recherche.

Ensuite, les utilisateurs ont également soulevé que les métadonnées des documents ne sont pas toujours mises en avant (notamment s’ils sont numérisés et, si c’est le cas, si ces documents sont entièrement numérisés). 
Les utilisateurs sont peu convaincus de la nécessité d’un compte utilisateur et les avantages qui y sont liés ne sont pas clairement énoncés. 
De façon plus superficielle, le graphisme n’est pas à la hauteur des internets modernes. 

Dernièrement, il apparait que les historiens que nous avons interrogés utilisent le site principalement pour la consultation des inventaires en ligne, utilisation peut-être fort limitée et non pas à la hauteur des capacités potentielles du site.
La posture hiératique des archivistes par rapport à la structure archivistique traditionnelle rend le site peu navigable pour les néophytes. Pouvoir outrepasser les limites de cette structure peut être réalisable de par les facultés techniques des moteurs de recherche contemporains. Les utilisateurs interviewés soulignent que des recherches de documents selon leur nature ou bien de façon transversale aux fonds d’archives seraient pertinentes.

\subsection{Les besoins et attentes des utilisateurs}
Nous avons choisi de représenter les besoins et attentes des utilisateurs au moyen d'une carte heuristique. Nous avons identifié quatre grands domaines : celui de la réduction du silence, du bruit, celui concernant l'aspect ergonomique du site, et enfin un domaine sur le fait d'outrepasser la structure archivistique, très complexe, nécessitant beaucoup de ressource de la part des utilisateurs. 
%insérer le mindmap sur les exigences fonctionnelles.

\subsection{Priorisation des besoins avec la matrice MoSCoW}
Dans l’état actuel, les besoins explorés sont en particulier ceux des historiens. Les retours sont très individualisés et complexes à réduire en des besoins univoques. Si l’on sait déjà ce qu’on cherche, le site est plus utilisable. Si l’on part un peu de rien et qu’on ne sait pas exactement et précisément ce qu’on cherche (début de constitution d’une problématique de recherche en histoire par exemple) le site s'avère très compliqué à utiliser. Les utilisateurs interrogés sont des historiens avec un master universitaire, qui utilisent le site et le connaissent bien, c’est la niche d’utilisateurs la plus habituée au site et ils ne composent qu'une petite partie des utilisateurs actuels et potentiels.
Les contraintes techniques ne peuvent pas être considérées puisque nous n’avons pas pu contacter les Archives de l’État et avoir accès à l’envers du décor en ce qui concerne l’infrastructure logicielle derrière le site web et le moteur de recherche.
%insérer la matrice MoSCoW

\section{Section 8 - Gestion du changement}
Nous avons dans cette partie considéré l'ensemble du projet de révision du site et du moteur de recherche des Archives de l'État, et non uniquement notre projet de récolte des besoins en vue de l'écriture du cahier des charges pour la révision du site. Cela faisait plus de sens de considérer l'ensemble du projet, puisque la gestion du changement ne s'applique pas à notre projet de récolte de besoins.

\subsection{Accompagnement du changement}
Pour élaborer ces stratégies d’accompagnement du changement, nous avons épousé au plus près le soutien au changement en évoquant les trois aspects de celui-ci : l'information, la compétence et la motivation.
Les archivistes ne sont pas des utilisateurs finaux dans le cadre de notre projet. Néanmoins, étant donné qu’ils sont les uniques référents à cet égard, il est primordial de les évoquer également dans cette partie. Ceci permet d’assurer la plus grande cohésion et la plus haute cohérence dans le changement induit par le projet.

Les personnes concernées par le changement seront informées de ce qui changera sur le site par l’installation d’une banderole et la présence d’un agenda succinct de la refonte et d’une FAQ. Le changement du site et de son moteur de recherche est pensé du point de vue des utilisateurs, ce qui n’avait pas été le cas jusqu’alors. Il est également pertinent de souligner les possibilités d’amélioration, qui ne sont pas toujours évidentes de prime abord. En effet, le changement provoque bien souvent de la résistance et fait craindre aux utilisateurs de perdre leurs habitudes. Une participation aux enquêtes de satisfaction et aux retours est fortement encouragée.
Les personnes seront compétentes grâce à l’appui et au secours du personnel des AGR formé sur la refonte du site. La FAQ viendra compléter cet appui pour ce qui est des questions plus spécifiques. Lors du travail de refonte, une attention particulière sera portée à la préservation de certains éléments graphiques du site pour susciter un sentiment de familiarité chez les utilisateurs actuels.
En ce qui concerne la motivation à accompagner le changement, les archivistes occupent un rôle important. Il convient de s’assurer de leur soutien et de leur bonne volonté pour convaincre à leur tour les utilisateurs. Un changement vers une meilleure intuitivité et plus d’ergonomie permet un accès facilité à de nouveaux utilisateurs (et pas seulement permettre à des historiens avec Master de savoir naviguer dans le site).

\subsection{Plan de communication}
Dans le cadre du projet global de refonte du site, il est évident que deux plans de communication doivent être élaborés : un plan interne et un plan externe. Étant donné le périmètre donné à notre travail, qui exclut notamment les archivistes, seul le plan externe, destiné à communiquer au public, fera ici l’objet d’une présentation.
Le plan comprend un unique groupe cible, correspondant aux utilisateurs finaux, parce que le plan de communication n’est pas différent selon les différents profils d’utilisateurs identifiés lors du business case.
%insérer le tableau du plan de communication

\section{Section 9 - Les méthodes de gestions de projet recommandées}
La méthode de gestion de projet que nous recommanderions est une méthode mixte entre Agile et Waterfall. Au vu du périmètre de notre projet et de la méconnaissance d’un certain nombre de paramètres (architecture informatique, aspect financier, etc.), il nous est difficile d’établir de propositions concrètes d’application de méthodes à des parties du projet. Néanmoins, nous avons considéré les deux grandes approches qui nous ont été présentées dans le cours et en avons établi la synthèse des forces et faiblesses de chacune dans le cadre spécifique de notre projet.

Des méthodes séquentielles proposées dans le cours, nous avons porté notre dévolu sur Prince2. Étant donné la nature informatique et l’aspect mouvant du projet, il faudrait une méthode souple qui nous permettrait de prendre en compte de nouveaux besoins (notamment induits par d'éventuelles évolutions techniques). En effet, Prince2 divise tout projet en étapes distinctes, sur lesquelles il est possible de revenir si le besoin s’en fait sentir. C’est cet aspect de la méthode qui rend possible la souplesse que nous cherchons.
En outre, le chef de projet est relativement libre dans son fonctionnement quotidien : il peut prendre tout un ensemble de décisions sans en référer aux autorités supérieures (management par exception). Néanmoins, il se doit de rendre des comptes au Comité de Pilotage du projet, à savoir dans notre cas la direction des Archives de l’État, ce qui est indispensable pour un projet qui utilise des fonds publics limités. Le Comité de Pilotage peut ainsi, à intervalles réguliers, remettre le projet sur le bon chemin s’il estime qu’il s’en écarte.
La méthode Prince2 a été initialement pensée pour une utilisation dans l’environnement gouvernemental britannique, ce qui la rend facilement adaptable à notre cas. En effet, Prince2 est une méthode souple et flexible, qui s’adapte aisément à un grand panel de projets. En outre, Prince2 s’adapte à des chefs de projet dont l’expérience est moindre, ce qui est toujours bienvenu.
À l’inverse, la méthode PMP, que nous avons choisi de ne pas privilégier, est une méthode complexe qui requiert des chefs de projet expérimentés et qui n’est pas souple, la rendant peu adaptée à notre cas. Néanmoins, tous les aspects de la méthode PMP ne sont pas à exclure, notamment le principe PCDA (basé sur la roue de Deming) qu’il serait tout à fait pertinent d’inclure conceptuellement parlant dans notre projet.

Les Archives de l’État sont une institution titanesque, avec une culture d’entreprise bien ancrée, qu’il est difficile de mettre en branle. Une méthode séquentielle est par conséquent la bienvenue. Néanmoins, et c’est une des raisons pour lesquelles nous avons décidé de privilégier Prince2 qui se combine facilement avec des méthodes Agile, cet aspect hiératique des Archives de l’État pourrait mener à de l’immobilisme que pourraient prévenir les méthodes Agiles précédemment mentionnées. 
L’Agilité forme une constellation de méthodes qui ne sont pas toutes indiquées pour notre projet. Par exemple, Lean Startup (mentionné dans le cours) ne nous semble guère approprié en raison de sa finalité : il n’est aucunement besoin de créer de nouvelles business ideas. En outre, le développement de solutions informatiques est particulièrement adapté aux méthodes Agile, qui ont vu le jour dans ce contexte.
L’existence, dans ces méthodes, d’un dialogue permanent entre l’ensemble des personnes impliquées dans le projet (équipe, responsables, etc.) est particulièrement précieux dans notre situation, pour favoriser un sentiment d’acceptation et d’enthousiasme dans le chef des archivistes qui pourraient être quelque peu réfractaires à l’idée de quelque changement. De plus, le fait que ces méthodes soient orientées “livrables” et qu’elles ne se formalisent en amont de tous les détails avant d’entamer le projet permet de contourner le problème de l’immobilisme précédemment évoqué. Des solutions (partielles) sont délivrées au terme de chaque itération (sprint de courte durée), soulignant la plus-value du projet entamé et  légitimant ce dernier tout du long.
Néanmoins, les méthodes Agile ne permettent pas à elles seules de se projeter sur le long terme, ce qui est un inconvénient majeur dans le cadre d’un projet relevant du public. Soulignons toutefois la souplesse de ces méthodes, qui permet de les utiliser en synergie avec d’autres méthodes.

En conclusion, nous préconiserions une fusion entre une méthode séquentielle (Prince2) et de la méthodologie Agile. Cette approche permettrait d’être particulièrement souple tout en maintenant un cadre solide et conducteur qui permet de justifier le projet aux yeux des utilisateurs internes comme externes.

\section{Retour sur expérience (REX)}
La méthode de gestion de projet que nous recommanderions est une méthode mixte entre Agile et Waterfall. Au vu du périmètre de notre projet et de la méconnaissance d’un certain nombre de paramètres (architecture informatique, aspect financier, etc.), il nous est difficile d’établir de propositions concrètes d’application de méthodes à des parties du projet. Néanmoins, nous avons considéré les deux grandes approches qui nous ont été présentées dans le cours et en avons établi la synthèse des forces et faiblesses de chacune dans le cadre spécifique de notre projet.

Des méthodes séquentielles proposées dans le cours, nous avons porté notre dévolu sur Prince2. Étant donné la nature informatique et l’aspect mouvant du projet, il faudrait une méthode souple qui nous permettrait de prendre en compte de nouveaux besoins (notamment induits par d'éventuelles évolutions techniques). En effet, Prince2 divise tout projet en étapes distinctes, sur lesquelles il est possible de revenir si le besoin s’en fait sentir. C’est cet aspect de la méthode qui rend possible la souplesse que nous cherchons.
En outre, le chef de projet est relativement libre dans son fonctionnement quotidien : il peut prendre tout un ensemble de décisions sans en référer aux autorités supérieures (management par exception). Néanmoins, il se doit de rendre des comptes au Comité de Pilotage du projet, à savoir dans notre cas la direction des Archives de l’État, ce qui est indispensable pour un projet qui utilise des fonds publics limités. Le Comité de Pilotage peut ainsi, à intervalles réguliers, remettre le projet sur le bon chemin s’il estime qu’il s’en écarte.
La méthode Prince2 a été initialement pensée pour une utilisation dans l’environnement gouvernemental britannique, ce qui la rend facilement adaptable à notre cas. En effet, Prince2 est une méthode souple et flexible, qui s’adapte aisément à un grand panel de projets. En outre, Prince2 s’adapte à des chefs de projet dont l’expérience est moindre, ce qui est toujours bienvenu.
À l’inverse, la méthode PMP, que nous avons choisi de ne pas privilégier, est une méthode complexe qui requiert des chefs de projet expérimentés et qui n’est pas souple, la rendant peu adaptée à notre cas. Néanmoins, tous les aspects de la méthode PMP ne sont pas à exclure, notamment le principe PCDA (basé sur la roue de Deming) qu’il serait tout à fait pertinent d’inclure conceptuellement parlant dans notre projet.

Les Archives de l’État sont une institution titanesque, avec une culture d’entreprise bien ancrée, qu’il est difficile de mettre en branle. Une méthode séquentielle est par conséquent la bienvenue. Néanmoins, et c’est une des raisons pour lesquelles nous avons décidé de privilégier Prince2 qui se combine facilement avec des méthodes Agile, cet aspect hiératique des Archives de l’État pourrait mener à de l’immobilisme que pourraient prévenir les méthodes Agiles précédemment mentionnées. 
L’Agilité forme une constellation de méthodes qui ne sont pas toutes indiquées pour notre projet. Par exemple, Lean Startup (mentionné dans le cours) ne nous semble guère approprié en raison de sa finalité : il n’est aucunement besoin de créer de nouvelles business ideas. En outre, le développement de solutions informatiques est particulièrement adapté aux méthodes Agile, qui ont vu le jour dans ce contexte.
L’existence, dans ces méthodes, d’un dialogue permanent entre l’ensemble des personnes impliquées dans le projet (équipe, responsables, etc.) est particulièrement précieux dans notre situation, pour favoriser un sentiment d’acceptation et d’enthousiasme dans le chef des archivistes qui pourraient être quelque peu réfractaires à l’idée de quelque changement. De plus, le fait que ces méthodes soient orientées “livrables” et qu’elles ne se formalisent en amont de tous les détails avant d’entamer le projet permet de contourner le problème de l’immobilisme précédemment évoqué. Des solutions (partielles) sont délivrées au terme de chaque itération (sprint de courte durée), soulignant la plus-value du projet entamé et  légitimant ce dernier tout du long.
Néanmoins, les méthodes Agile ne permettent pas à elles seules de se projeter sur le long terme, ce qui est un inconvénient majeur dans le cadre d’un projet relevant du public. Soulignons toutefois la souplesse de ces méthodes, qui permet de les utiliser en synergie avec d’autres méthodes.

En conclusion, nous préconiserions une fusion entre une méthode séquentielle (Prince2) et de la méthodologie Agile. Cette approche permettrait d’être particulièrement souple tout en maintenant un cadre solide et conducteur qui permet de justifier le projet aux yeux des utilisateurs internes comme externes.







\end{document}
